\begin{abstract}

 \begin{otherlanguage*}{portuguese}
   
    \devops mudou a indústria de software para permitir a entrega contínua.
    Embora muitos estudos tenham investigado como introduzir \devops em um
    produto de software do ponto de vista organizacional, menos se sabe sobre os
    desafios técnicos que os desenvolvedores e profissionais enfrentam ao
    transformar códigos legados em DevOps, apesar da importância indiscutível
    deste tópico. Neste artigo, através do contexto de aplicações web, relatamos
    os resultados de um estudo de caso com a adoção de quatrp projetos legados
    de código aberto em \devops para entender quais técnicas e estratégias de
    refatoração influenciam as decisões dos desenvolvedores. Analisamos duas
    variáveis ​​dependentes: a técnica usada e como são aplicadas ao projeto.
    Após cada implementação, havia uma visão geral do processo acabado de
    ocorrer e posteriormente um relatório escrito sobre como as estratégias
    foram aplicadas, sua respectiva ordem, qual estratégia foi mais vantajosa e
    afins. Esses relatórios foram a base deste estudo. As principais conclusões
    desse estudo são que algumas estratégias são mais eficientes quando vistas
    do ponto de vista da evolução e a sequência em que essas técnicas são
    empregadas importa.

   \vspace{\onelineskip}
 
   \noindent 
   \textbf{Key-words}: Devops, Refatoração, Compreensão de Programa, Relatório de Experiência, Guia, Legado.
 \end{otherlanguage*}
\end{abstract}
